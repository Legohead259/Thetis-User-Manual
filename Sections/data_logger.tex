\section{Data logger}
\label{sec:dataLogger}

Thetis can function as a stand-alone data logger by streaming real-time data to a file on the \ac{microSD}.
Files created by the data logger use the .bin extension and can be downloaded from the device to be converted to \ac{CSV} files using the product software.

The data logger will create a new file in the root directory on the \ac{microSD} each time the device boots.
Files will never be overwritten or deleted by the data logger.
If the \ac{microSD} becomes full then the data logger will stop and Thetis will indicate an error.

\subsection{Start and stop}

The data logger is started or stopped by the ``Log Enable'' button (\seeSection{sec:buttons}).
As soon as the button is held for half of a second, the device will immediately begin logging.
When the operator wishes to stop logging, the ``Log Enable'' button can be held again for another half second or until the ``Activity'' LED and \acs{RGB} \acs{LED} reflect the ``Ready'' state.

\vskip 3em

\warning{The data file is not written to the microSD card filesystem until the device is commanded to stop logging.
DO NOT reset or power off the device when logging, otherwise the data may not be saved or will be corrupted!}

\subsection{File name}
\label{sec:fileName}

The file name format is ``log\textunderscore CCC.bin'' where ``CCC'' is a counter.
The counter runs between ``000'' and ``999''.
On startup, Thetis scans the microSD card and increments the counter until the next available number is encountered.
When it exceeds its maximum value, the system will throw a filesystem initialization error.
Even if the device is powered on, but does not log, a new log file will be created.
In testing environments, this may lead to multiple empty files that can obscure an actual desired data file.

\subsection{File contents}

The contents of the file is a byte stream as per the communication protocol described in \Fref{sec:communicationProtocol}.
