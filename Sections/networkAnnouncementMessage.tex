\section{Network announcement message}

The network announcement message is used by a host to discover and connect to devices on the same network.  The message is continuously broadcast by the device on \ac{UDP} port 10000 at a fixed rate of one message per second.  Each message provides the device name, serial number, Wi-Fi and battery status, as well as the device settings required for a host to establish a \ac{TCP} or \ac{UDP} connection.  The message is a single \ac{JSON} object.  The key/value pairs are described in \Fref{tab:networkAnnouncementMessage}.

\customTable
{l l l}
{Key & Value type & Description}
{
    \enquote{sync} & number & Used for synchronisation\\
    \enquote{name} & string & Device name\\
    \enquote{serial} & string & Device serial number\\
    \enquote{ip} & string & Device \acs{IP} address\\
    \enquote{port} & number & \acs{TCP} port\\
    \enquote{send} & number & \acs{UDP} send port (device sends to this port)\\
    \enquote{receive} & number & \acs{UDP} receive port (device receives on this port)\\
    \enquote{rssi} & number & \acs{RSSI} percentage (always 0 in Wi-Fi AP mode)\\
    \enquote{battery} & number & Battery percentage\\
    \enquote{status} & number & Charging status (See \Fref{tab:chargingStatusEnumeration})\\
}
{Network announcement message key/value pairs}
{tab:networkAnnouncementMessage}

\begin{table}[H]
    \begin{tabular}{l l l}
        \textbf{Example}*\textbf{:} & \texttt{\{}\\
        & \texttt{~~"sync":} & \texttt{0,}\\
        & \texttt{~~"name":} & \texttt{"x-IMU3",}\\
        & \texttt{~~"serial":} & \texttt{"0123-4567-89AB-CDEF",}\\
        & \texttt{~~"ip":} & \texttt{"192.168.1.1",}\\
        & \texttt{~~"port":} & \texttt{7000,}\\
        & \texttt{~~"send":} & \texttt{8000,}\\
        & \texttt{~~"receive":} & \texttt{9000,}\\
        & \texttt{~~"rssi":} & \texttt{100,}\\
        & \texttt{~~"battery":} & \texttt{100,}\\
        & \texttt{~~"status":} & \texttt{2}\\
        & \texttt{\}}
    \end{tabular}\\
    \begin{tabular}{l}
        \\
        \footnotesize{* The actual \acs{JSON} will not include any whitespace.}
    \end{tabular}
\end{table}
