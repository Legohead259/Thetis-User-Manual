\section{Technical specification}

\newcommand{\characteristicTable}[4]{
    \customTable
    {l c c}
    {Characteristic & Value & Notes}
    {
        #1
    }
    {#2}
    {#3}
    \textbf{Notes}
    \begin{enumerate}[nolistsep]
        #4
    \end{enumerate}
}

\subsection{Temperature}
\label{sec:temperature}

\newcommand{\noteHeatGenerated}{The temperature of the device will always be greater than the surroundings due to heat generated by electronics.}

\newcommand{\noteCalibrationTemperature}{The specified accuracy of the device is not achieved over the full operating temperature range.}% See \Fref{sec:calibration} for more information.}

\subsubsection{No battery}

\characteristicTable
{
    Operating & -40\textdegree{}C to 85\textdegree{}C & \ref{itm:temperatureNoBattery1}, \ref{itm:temperatureNoBattery2}\\
    Storage & -40\textdegree{}C to 105\textdegree{}C & -\\
}
{Temperature specification (no battery)}
{tab:temperatureSpecificationNoBattery}
{
    \item \label{itm:temperatureNoBattery1} \noteHeatGenerated
    \item \label{itm:temperatureNoBattery2} \noteCalibrationTemperature
}

\subsubsection{With battery}

\characteristicTable
{
    Operating (discharging) & -20\textdegree{}C to 60\textdegree{}C & \ref{itm:temperatureWithBattery1}, \ref{itm:temperatureWithBattery2}\\
    Operating (charging) & 0\textdegree{}C to 45\textdegree{}C & \ref{itm:temperatureWithBattery1}, \ref{itm:temperatureWithBattery2}, \ref{itm:temperatureWithBattery3}\\
    Storage & -20\textdegree{}C to 25\textdegree{}C & -\\
}
{Temperature specification (with battery)}
{tab:temperatureSpecificationWithBattery}
{
    \item \label{itm:temperatureWithBattery1} \noteHeatGenerated
    \item \label{itm:temperatureWithBattery2} \noteCalibrationTemperature
    \item \label{itm:temperatureWithBattery3} Charging at temperatures below 0\textdegree{}C will reduce the capacity and cycle life of the battery.
}

\subsection{Sensors}

\newcommand{\noteSampleRate}{Each sample includes a timestamp for a reliable measurement of time independent of the sample rate error.  See \Fref{sec:sampleRatesMessageRatesAndTimestamps} for more information.}

\newcommand{\noteBandwidth}{The maximum bandwidth is achieved when the message rate is equal to the sample rate.  If the message rate is less than the sample rate then samples are averaged.  See \Fref{sec:sampleRatesMessageRatesAndTimestamps} for more information.}

\newcommand{\noteAccuracy}[3]{The accuracy at 1 #1 is evaluated as the deviation of the measured magnitude of #2 for a 360\textdegree{} rotation around the X, Y, and Z axis aligned to the #3 axis.}

\subsubsection{Gyroscope}

\characteristicTable
{
    Range & \textpm{}2000\textdegree{}/s & -\\
    Resolution & 16-bit, 0.061\textdegree{}/s & -\\
    Sample rate & 400 Hz \textpm{}0.3\% & \ref{itm:gyroscope1}\\
    Bandwidth & 47 Hz & \ref{itm:gyroscope2}\\
    Noise & 0.014\textdegree{}/s/√Hz & -\\
}
{Gyroscope specification}
{tab:gyroscopeSpecification}
{
    \item \label{itm:gyroscope1} \noteSampleRate
    \item \label{itm:gyroscope2} \noteBandwidth
}

\subsubsection{Accelerometer}

\characteristicTable
{
    Range & \textpm{}24 g & -\\
    Resolution & 16-bit, 732 \textmugreek{}g & -\\
    Sample rate & 400 Hz \textpm{}0.3\% & \ref{itm:accelerometer1}\\
    Bandwidth & 145 Hz & \ref{itm:accelerometer2}\\
    Noise & 160 \textmugreek{}g/√Hz (X, Y), 190 \textmugreek{}g/√Hz (Z) & -\\
    Accuracy at 1 g & TBC & \ref{itm:accelerometer3}\\
}
{Accelerometer specification}
{tab:accelerometerSpecification}
{
    \item \label{itm:accelerometer1} \noteSampleRate
    \item \label{itm:accelerometer2} \noteBandwidth
    \item \label{itm:accelerometer3} \noteAccuracy{g}{gravity}{horizontal}
}

\subsubsection{Magnetometer}

\characteristicTable
{
    Range & \textpm{}1300 \textmugreek{}T (X, Y), \textpm{}2500 \textmugreek{}T (Z) & -\\
    Sample rate & 20 Hz \textpm{}8\% & \ref{itm:magnetometer1}\\
    Noise & 0.3 \textmugreek{}T & -\\
    Accuracy at 1 \acs{a.u.} & TBC & \ref{itm:magnetometer2}, \ref{itm:magnetometer3}\\
}
{Magnetometer specification}
{tab:magnetometerSpecification}
{
    \item \label{itm:magnetometer1} \noteSampleRate
    \item \label{itm:magnetometer2} The calibrated magnetometer units are \ac{a.u.}.  1 \ac{a.u.} is equal to the magnitude of the ambient magnetic field during calibration, approximately 500 \textmugreek{}T.
    \item \label{itm:magnetometer3} \noteAccuracy{\ac{a.u.}}{the ambient magnetic field}{vertical}
}

\subsubsection{High-g accelerometer}

\characteristicTable
{
    Range & \textpm{}200 g & -\\
    Resolution & 16-bit, 6.1 mg & -\\
    Sample rate & 1600 Hz \textpm{}2\% & \ref{itm:highGAccelerometer1}\\
    Bandwidth & 800 Hz & \ref{itm:highGAccelerometer2}\\
    Noise & 5 mg/√Hz & -\\
    Accuracy at 1 g & TBC & \ref{itm:highGAccelerometer3}\\
}
{High-g Accelerometer specification}
{tab:highGAccelerometerSpecification}
{
    \item \label{itm:highGAccelerometer1} \noteSampleRate
    \item \label{itm:highGAccelerometer2} \noteBandwidth
    \item \label{itm:highGAccelerometer3} \noteAccuracy{g}{gravity}{horizontal}
}

\subsubsection{Temperature sensor}

\characteristicTable
{
    Range & -104\textdegree{}C to 150\textdegree{}C & \ref{itm:temperatureSensor1}\\\
    Sample rate & 5 Hz \textpm{}0.3\% & \ref{itm:temperatureSensor2}, \ref{itm:temperatureSensor3}\\
    Accuracy & \textpm{}1\textdegree{}C at 25\textdegree{}C & -\\
}
{Temperature sensor specification}
{tab:temperatureSensor}
{
    \item \label{itm:temperatureSensor1} The temperature sensor measurement range exceeds the device operating temperature range.  See \Fref{sec:temperature} for more information.
    \item \label{itm:temperatureSensor2} \noteSampleRate
    \item \label{itm:temperatureSensor3} The temperature sensor is oversampled and the result decimated to the specified sample rate.
}

% \subsection{\acs{AHRS}}

% ...

% \subsection{Latency}

% ...

% \subsection{Wireless throughput}

% ...

% \subsection{Battery life}

% \subsubsection{Data logger}

% ...

% \subsubsection{W-Fi client 2.4 GHz}

% ...

% \subsubsection{W-Fi client 5 GHz}

% ...

% \subsubsection{W-Fi AP 2.4 GHz}

% ...

% \subsubsection{W-Fi AP 5 GHz}

% ...

% \subsection{Example message rates}

% Some aspect of performance is often dependent on message rates.

% % Performance is dependent on send rates... Here are some exmaples
% % - Default message rate
% % - IMU maximum
% % - All maximum

% % binary

% \customTable
% {l c c c}
% {Message type & Default & IMU maximum & All maximum}
% {
%     Inertial (gyroscope and accelerometer) & 50 & 400 & 400\\
%     Magnetometer & 20 & 20 & 20\\
%     AHRS (quaternion) & 50 & 400 & 400\\
%     Temperature & 1 & 1 & 5\\
%     High-g & 50 & 0 & 1600\\
%     Battery & 1 & 1 & 5\\
%     Wi-Fi \acs{RSSI} & 0 & 0 & 1\\
% }
% {Example message rates (messages/s)}
% {tab:exampleMessageRates}

% \subsection{Data logging capacity}

% \Fref{tab:dataLoggingCapacity} summaries the data logging capacity ... continuous logging... for example message rates described in \Fref{tab:exampleMessageRates}.

% \customTable
% {l c c}
% {Message rates & 8 GB & 32 GB}
% {
%     Default & 485 hours & 1941 hours\\
%     IMU maximum & 91 hours & 363 hours\\
%     All maximum & 37 hours & 149 hours\\
% }
% {Data logging capacity}
% {tab:dataLoggingCapacity}
