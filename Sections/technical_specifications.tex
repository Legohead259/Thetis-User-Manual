\section{Technical Specification}

\newcommand{\techincalTable}[5]{
    \customTable
    {l c c}
    {#1 & Value & Notes}
    {
        #2
    }
    {#3}
    {#4}
    \textbf{Notes}
    \begin{enumerate}[nolistsep]
        #5
    \end{enumerate}
}

\newcommand{\characteristicTable}[4]{
    \techincalTable
    {Characteristic}
    {#1}
    {#2}
    {#3}
    {#4}
}

\newcommand{\conditionTable}[4]{
    \techincalTable
    {Condition}
    {#1}
    {#2}
    {#3}
    {#4}
}

\subsection{Mechanical}

\subsubsection{Board}

\newcommand{\noteMechanicalDrawings}[1]{A detailed mechanical drawing describing the #1 dimensions and locations of key components is available in the Appendix.}

\characteristicTable
{
    Size & 1.82 $\times$ 1.74 $\times$ 0.48 in & \ref{itm:boardMechanicalSpecification1}\\
    Weight & 3.2 g & -\\
}
{Board mechanical specification}
{tab:boardMechanicalSpecification1}
{
    \item \label{itm:boardMechanicalSpecification1} \noteMechanicalDrawings{board}
}

\characteristicTable
{
    Size & 3.15 $\times$ 2.36 $\times$ 0.79 in & \ref{itm:housingMechanicalSpecification1}\\
    Weight & 50 g & -\\
}
{Housing mechanical specification}
{tab:housingMechanicalSpecification1}
{
    \item \label{itm:housingMechanicalSpecification1} \noteMechanicalDrawings{housing}
}

\subsection{Temperature}
\label{sec:temperature}

\newcommand{\noteHeat}{The operating temperature of the device will always be greater than the surroundings due to heat generated by electronics.}

\newcommand{\noteFullRange}{The specified accuracy of the device is not achieved over the full operating temperature range.  \seeSection{sec:calibration}}

\subsubsection{No battery}

\characteristicTable
{
    Operating & -40\textdegree{}C to 85\textdegree{}C & \ref{itm:temperatureNoBattery1}, \ref{itm:temperatureNoBattery2}\\
    Storage & -40\textdegree{}C to 105\textdegree{}C & -\\
}
{Temperature specification (no battery)}
{tab:temperatureSpecificationNoBattery}
{
    \item \label{itm:temperatureNoBattery1} \noteHeat
    \item \label{itm:temperatureNoBattery2} \noteFullRange
}

\subsubsection{With battery}

\characteristicTable
{
    Operating (discharging) & -20\textdegree{}C to 60\textdegree{}C & \ref{itm:temperatureWithBattery1}, \ref{itm:temperatureWithBattery2}\\
    Operating (charging) & 0\textdegree{}C to 45\textdegree{}C & \ref{itm:temperatureWithBattery1}, \ref{itm:temperatureWithBattery2}, \ref{itm:temperatureWithBattery3}\\
    Storage & -20\textdegree{}C to 25\textdegree{}C & -\\
}
{Temperature specification (with battery)}
{tab:temperatureSpecificationWithBattery}
{
    \item \label{itm:temperatureWithBattery1} \noteHeat
    \item \label{itm:temperatureWithBattery2} \noteFullRange
    \item \label{itm:temperatureWithBattery3} Charging at temperatures below 0\textdegree{}C will reduce the capacity and cycle life of the battery.
}

\subsection{Sensors}

\newcommand{\noteRate}[1]{Each #1 includes a timestamp for a reliable measurement of time independent of the #1 rate error.  \seeSection{sec:sampleRatesMessageRatesAndTimestamps}}

\newcommand{\noteBandwidth}{The maximum bandwidth is achieved when the message rate is equal to the sample rate.  If the message rate is less than the sample rate then samples are averaged.  \seeSection{sec:sampleRatesMessageRatesAndTimestamps}}

\newcommand{\noteAccuracy}[3]{The #1 error is evaluated as the deviation of the measured magnitude of #2 for a 360\textdegree{} rotation around each axis aligned to the #3.  The magnitude is calculated as $\sqrt{x^2 + y^2 + z^2}$.}

\newcommand{\noteTemperature}{Accuracy is specified for the calibrated temperature only. \seeSection{sec:calibration}}

\subsubsection{Gyroscope}

\characteristicTable
{
    Range & \textpm{}2000\textdegree{}/s & -\\
    Resolution & 16-bit, 0.061\textdegree{}/s & -\\
    Sample rate & 100 Hz \textpm{}0.3\% & \ref{itm:gyroscope1}\\
}
{Gyroscope specification}
{tab:gyroscopeSpecification}
{
    \item \label{itm:gyroscope1} \noteRate{sample}
}

\subsubsection{Accelerometer}

\characteristicTable
{
    Range & \textpm{}32 g & -\\
    Resolution & 16-bit, 488 \textmugreek{}g & -\\
    Sample rate & 100 Hz \textpm{}0.3\% & \ref{itm:accelerometer1}\\
    Accuracy at 1 g & \textpm{}5 mg & \ref{itm:accelerometer3}, \ref{itm:accelerometer4}\\
}
{Accelerometer specification}
{tab:accelerometerSpecification}
{
    \item \label{itm:accelerometer1} \noteRate{sample}
    \item \label{itm:accelerometer3} \noteAccuracy{accelerometer}{gravity}{horizontal}
    \item \label{itm:accelerometer4} \noteTemperature
}

\subsubsection{Magnetometer}

\characteristicTable
{
    Range & \textpm{}16 Gauss & -\\
    Sample rate & 80 Hz \textpm{}8\% & \ref{itm:magnetometer1}\\
    Noise & Unknown & -\\
    Accuracy at 1 \acs{a.u.} & Unknown & \ref{itm:magnetometer2}, \ref{itm:magnetometer3}, \ref{itm:magnetometer4}\\
}
{Magnetometer specification}
{tab:magnetometerSpecification}
{
    \item \label{itm:magnetometer1} \noteRate{sample}
    \item \label{itm:magnetometer2} The calibrated magnetometer units are \ac{a.u.}.  1 \ac{a.u.} is equal to the magnitude of the ambient magnetic field during calibration, approximately 45 \textmugreek{}T.
    \item \label{itm:magnetometer3} \noteAccuracy{magnetometer}{the ambient magnetic field}{vertical}
    \item \label{itm:magnetometer4} \noteTemperature
}

\subsection{\acs{AHRS}}

\subsubsection{Update rate}

\characteristicTable
{
    Update rate & 64 Hz \textpm{}0.3\% & \ref{itm:ahrsUpdateRate1}, \ref{itm:ahrsUpdateRate2}\\
}
{\acs{AHRS} update rate}
{tab:ahrsUpdateRate}
{
    \item \label{itm:ahrsUpdateRate1} \noteRate{update}
    \item \label{itm:ahrsUpdateRate2} The \ac{AHRS} update rate is fixed independent of message rate settings.  \ac{AHRS} outputs are not averaged when the message rate is less than the update rate.
}

\subsubsection{Static accuracy}

\characteristicTable
{
    Inclination & 0.5\textdegree{} \acs{RMS} & \ref{itm:ahrsStaticAccuracy1}, \ref{itm:ahrsStaticAccuracy2}\\
    Heading & 1\textdegree{} \acs{RMS} & \ref{itm:ahrsStaticAccuracy1}, \ref{itm:ahrsStaticAccuracy2}\\
}
{\acs{AHRS} static accuracy}
{tab:ahrsStaticAccuracy}
{
    \item \label{itm:ahrsStaticAccuracy1} Static accuracy is specified as the \ac{RMS} error for a 360\textdegree{} rotation around each axis.
    \item \label{itm:ahrsStaticAccuracy2} \noteTemperature
}

\subsection{Data logger capacity}

\newcommand{\noteBinary}{The data logging capacity is specified for binary data messages.  Capacity will be reduced for \acs{ASCII} data messages.}

The data logger capacity can be determined with the following equation:

\begin{equation} \labeq{storage_time}
    t_{\text{samples}} = \frac{N_{\text{storage}} [\text{Bytes}]}{198 [\text{Bytes}] \times 64 [\text{s}^{-1}] \times 3600 \left[\frac{\text{s}}{\text{h}}\right]}
\end{equation}

This yields the following times the data logger can record for with a given \acs{microSD} card size:

\customTable
{l c}
{Size & Value}
{
    1 GB & 22 hours \\
    4 GB & 88 hours \\
    8 GB & 175 hours \\
    16 GB & 350 hours \\
    32 GB & 701 hours \\
}
{Data logger record times based on capacity of the \acs{microSD} card}
{tab:data_logger_times}
